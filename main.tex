\documentclass[12pt]{article}

\title{Ficha Circunferência}
\author{Tomás Pereira}

\begin{document}
\maketitle

1.\\\\
$O\hat{C}A=28^{\circ}$\\
$C\hat{O}A=180^{\circ}-2(28^{\circ})=180^{\circ}-56^{\circ}=124^{\circ}$\\
$C\hat{B}A=\frac{124^{\circ}}{2}=62^{\circ}$\\

R: $62^{\circ}$.\\



2.\\\\
$C\hat{D}E=\frac{180^{\circ}-70^{\circ}}{2}=55^{\circ}$\\
$C\hat{E}D=55^{\circ}$\\
ArcoBC = $\frac{50^{\circ}}{2}=25^{\circ}$\\
ArcoBD = $55^{\circ}\cdot2-25^{\circ}=110^{\circ}-25^{\circ}=85^{\circ}$\\

R: $85^{\circ}$.\\


3.\\\\
$180^{\circ}-100^{\circ}=80^{\circ}$\\
ArcoBA = $80^{\circ}\cdot2=160^{\circ}$\\
ArcoBCA = $360^{\circ}-160^{\circ}=200^{\circ}$\\
R: $200^{\circ}$.\\\\\\

4.\\\\
$B\hat{C}E=\frac{60^{\circ}}{2}=30^{\circ}$\\
$C\hat{B}E=180^{\circ}-30^{\circ}-90^{\circ}=60^{\circ}$\\
ArcoCD = $60^{\circ}\cdot2=120^{\circ}$\\\\
R: B.\\

5.\\\\
$D\hat{B}C=\frac{110^{\circ}}{2}=55^{\circ}$\\
$D\hat{C}B=D\hat{B}C$\\
$B\hat{D}C=180^{\circ}-2(55^{\circ})=180^{\circ}-110^{\circ}=70^{\circ}$\\\\
R: A.\\

6.\\\\
$ArcoCB \equiv ArcoBA \equiv ArcoAE \equiv ArcoDE \equiv ArcoDC$\\
ArcoCB = $\frac{360^{\circ}}{2}=72^{\circ}$\\
$C\hat{A}J \equiv A\hat{C}J$\\
$C\hat{A}J=180^{\circ}-2(36^{\circ})=180^{\circ}-72^{\circ}=108^{\circ}$\\\\
R: CJA tem 108º de amplitude.\\

7.\\\\
$D\hat{E}A=130^{\circ}$\\
ArcoCD  = $180^{\circ}-130^{\circ}=50^{\circ}$\\
$D\hat{E}C=\frac{50^{\circ}}{2}=25^{\circ}$\\\\
R: DEC tem 25º de amplitude.




\end{document}
